\documentclass[letterpaper,12pt]{article}
\usepackage{tabularx,amsmath,boxedminipage,graphicx}
\usepackage[margin=1in,letterpaper]{geometry} % this shaves off default margins which are too big
\usepackage{cite}
\usepackage[utf8]{inputenc}
\usepackage{polski}
\usepackage{indentfirst}
\usepackage{float}
\usepackage{mathtools}
\usepackage{amssymb}
\usepackage{array}
\usepackage{tabu}
\usepackage{booktabs}
\usepackage{listings}
\usepackage[final]{hyperref} % adds hyper links inside the generated pdf file
\hypersetup{
	colorlinks=true,       % false: boxed links; true: colored links
	linkcolor=black,          % color of internal links
	citecolor=blue,        % color of links to bibliography
	filecolor=magenta,      % color of file links
	urlcolor=blue         
}
\lstset{
    basicstyle=\ttfamily\footnotesize,
    frame=single, % adds a frame around the code
    xleftmargin=3.4pt,
    xrightmargin=3.4pt,
    breaklines=true
}
\begin{document}
Gra odbywa się w systemie turowym. Ekran rozgrywki składa się z heksagonalnej planszy oraz jednostek graczy.
Walka jest podzielona na rundy i tury. Tura jest synonimem poruszenia się pojedyncza jednostka. Podczas jednej rundy akcja musi zostać podjęta wobec każdej jednostki. Ruchy nie są wykonywane na przemian tak jak na przykład w szachach, lecz ich ruchy są uzależnione od cech posiadanych jednostek. Gracze nie wybierają jednostek który poruszają w danych turze - kolejność jednostek zależna od ich cech. Rozgrywka nie ma ograniczonej ilości rund. Warunkiem zakończenie walki jest wyeliminowanie wszystkich wrogich jednostek.
Możliwe akcje jednostki podczas tury:
\begin{itemize}  
\item Ruch 
\par Jednostka może wykorzystać swoja turę na przemieszczanie się pomiędzy swoim aktualnym polem, a polem znajdującym się w zasięgu jej ruchu. Ruch odbywa się po najkrótszej ścieżce znalezionej pomiędzy danymi polami. Ruch nie może się obyć poza plansze, pole zajęte przez inną jednostkę, pole zajęte przez przeszkodę. Jeżeli jej umiejętność nie definiuje innego sposobu przemieszczania to jednostka nie może przenikać przez inne jednostki oraz przeszkody. Wykorzystanie tej akcji kończy turę. 
\item Atak
\par Jeżeli w zasięgu ataku znajduję się jednostka wroga do gracza, może zostać podjęta akcja ataku (opisane w paragrafie poświęconemu atakowi). Wykorzystanie tej akcji kończy turę. 
\item Ruch i Atak
\par Jeżeli zasięg jednostki pozwala na przesunięcie tak aby wroga jednostka znajdowała się w zasięgu ataku po poruszeniu obie te akcje można połączyć w jedną.
\item Użycie umiejętności 
\par Cześć jednostek posiada umiejętności które mogą zostać wykorzystane podczas tury danej jednostki  (opisane w paragrafie poświęconemu umiejętnością). Wykorzystanie tej akcji kończy turę. 
\item Obrona
\par Tura może zostać wykorzystanie do nadania jednostce statu ''Obrona''. Jednostka z tym statusem otrzymuje obrażenia zmniejszone o połowę, aż do swojej następnej tury. Wykorzystanie tej akcji kończy turę. 
\item Czekanie
\par Akcja czekanie przenosi turę jednostki na koniec aktualnej kolejki jednostek. Oznacza to że akcja jednostki zostanie podjęta jeszcze raz podczas tej samej rundy. Akcja ta może zostać wykorzystana raz na rundę (Jeżeli w danej rundzie jednostka czekała nie może czekać jeszcze raz aż do następnej tury) 
\end{itemize}
 


\end{document}